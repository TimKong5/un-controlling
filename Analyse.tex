% !TEX root =  master.tex
\chapter{Kategorisierung und Analyse der integrierten Berichtwerkzeuge }
Im folgenden Kapitel wird die im Zusammenhang der vorliegenden Arbeit durchgeführte Anforderungsanalyse vorgestellt. Diese steht im fachlichen Kontext eines Versicherungs-Service-Centers und stellt die Anforderungen an eine softwaretechnische Unterstützung eines Service-Center-Agenten im Bereich des Kundendatenmanagements dar.

\section{Kriterien als Grundlage der Kategorisierung}
Grundlage der Anforderungsanalyse bildet die Ermittlung der vorhandenen Stakeholder. Diese sind alle internen und externen Interessensgruppen, die durch den Einsatz der Software beeinträchtigt sind. 
Die Abbildung \ref{stakeholder} zeigt eine Matrix, welche die wichtigsten Stakeholder und zugleich eine Einordnung in die Faktoren Einflusskraft und generellem Interesse am Projekt beinhaltet.
Dies ist vor allem für den Projektleiter von hoher Wichtigkeit, da sich daraus unter anderem die für ihn relevanten Prioritäten der Stakeholder ablesen lassen.
\\
\begin{figure}[h]
	\begin{center}
		\includegraphics[width=10cm]{img/Stakeholder.png},
		\caption[Stakeholder-Matrix]
		{Stakeholder-Matrix (Eigene Abbildung)}
		\label{stakeholder}
	\end{center}
\end{figure}
\\
Wie es bei den meisten Projekten üblich ist, hat das Management auch bei der softwaretechnischen Unterstützung eines Versicherungs-Service-Centers den größten Einflussfaktor.  Dies beruft sich vor allem auf die Abhängigkeit des Projektes von deren Genehmigung.  Das Problem hierbei ist meist jedoch das beschränkte Interesse, welches das Management dem Projekt entgegenbringt. Um dies revidieren zu können, müssen klare finanzielle oder ökologische Vorteile für das Unternehmen aufgezeigt werden. \\
Ein weiterer Stakeholder ist der Services-Center-Agent. Dieser hat als normale Arbeitskraft generell sehr wenig Einfluss auf den Verlauf des Projektes. Jedoch verfügt er über ein großes Interesse, da er zum einen der Hauptbenutzer der Software ist und täglich damit zurechtkommen muss. 
\newpage
Zum anderen haben die positiven Ergebnisse, die dieser mittels seines verbesserten Arbeitsumfeldes erreichen kann, einen generellen Einfluss auf das gesamte Unternehmen. 
Deshalb sollte er trotz seines geringen Einflusses mit in die Anforderungsanalyse einer softwaretechnischen Unterstützung und der generellen Entwicklung eines solchen Systems einbezogen werden. \\
\\
Zudem gibt es neben direkten Betroffenen auch indirekt Betroffene, wie es beispielsweise die Verkaufs- und Marketingabteilung sind. Diese werden im Allgemeinen weniger Kontakt mit der Software haben, profitieren jedoch von den daraus gewonnenen Daten und Informationen. Hierbei hat normalerweise die Marketingabteilung geringfügig mehr Nutzen als die Verkaufsabteilung, da diese mehr kundenspezifische Informationen benötigen. Jedoch ist es in den meisten Unternehmen so, dass die Verkaufsabteilung auf Grund der wirtschaftlichen Verantwortlichkeit einen höheren Stellenwert und somit auch einen höheren Einflussfaktor, als die Marketingabteilung, hat. 

Der Administrator ist primär für die Implementierung der Software zuständig. Deshalb verfügt er grundlegend eher über ein geringfügiges Interesse und einem unterdurchschnittlichen Einflussfaktor. 
\newpage
Der Kontakt des Kunden mit der Service-Center-Software beschränkt sich ausschließlich auf indirekte Wege. Diese profitiert von dem, durch die Software verbesserten, Service. Wie jedoch bereits im ersten Kapitel beschrieben, setzen die Unternehmen auf die ``Customer-Driven Growth Revolution``, weshalb der Kunde an sich einen hohen Einflussfaktor hat. Zudem hat fast jeder Kunde ein Interesse an einem zufriedenstellenden, sowie auch einem effizienten Service. 
\section{Personas der integrierten Berichtwerkzeuge von S/4HANA}

\section{Analyse und Bewertung mittels einer Matrix}
In diesem Abschnitt werden die erarbeiteten Anforderungen der zuvor erläuterten Interessensgruppen vorgestellt. Jedoch wird hierbei, um den Rahmen der Projektarbeit 1 nicht zu sprengen, eine Beschränkung auf das Management, den Service-Center-Agenten und den Administrator durchgeführt.\\
Die generellen Anforderungen der Stakeholder an ein Dashboard für den Service-Center-Agenten lassen sich aus dem Use-Case-Diagramm in der Abbildung \ref{use-case-diagramm} entnehmen. Die in dieser Grafik dargestellten Anforderungen werden im Kapitel 3.3 spezifiziert. Doch zuerst werden diese im nächsten Kapitel strukturiert und auf gegenseitige Zielkonflikte untersucht.\\
\begin{figure}[h]
	\begin{center}
		\includegraphics[width=16cm]{img/Versicherungs-Service-Center.png},
		\caption[Use-Case-Diagramm Versicherungs-Service-Center]
		{Use-Case-Diagramm Versicherungs-Service-Center \\(Eigene Abbildung)}
		\label{use-case-diagramm}
	\end{center}
\end{figure}
 
