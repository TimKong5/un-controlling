% !TEX root =  master.tex
\chapter{Einleitung}
\section{Motivation}
Wie bereits eine Studie des Bundesministeriums für Wirtschaft und
Technologie aus dem Jahr 2010 aufzeigt und prognostiziert hat, stieg die Popularität von \ac{SaaS}-Lösungen innerhalb der letzten Jahre und soll bereits im Jahr 2025 90\% der Gesamtausgaben für Standardsoftware in Deutschland betragen.\autocite[Vgl.][S. 58]{Dufft.2010}\\
Dabei hat sich besonders im Cloud nativen Umfeld die containerbasierte Bereitstellung der \ac{SaaS}-Lösung mit der Containertechnologie Docker und der Kubernetes Plattform zur Orchestration der Docker Container etabliert.\\
Auch das Projekt SAP Subscription Billing zielt auf die Entwicklung und den Betrieb einer Cloud nativen \ac{SaaS}-Lösung ab, weshalb sich das Architektenteam bei der Softwarearchitektur für eine auf Microservices basierende Architektur entschieden hat. Für die Bereitstellung verwendet das Projekt die von der \ac{SCP} angebotene \ac{CF}-Umgebung.\\ 
Allerdings zeigte sich bei der Entwicklung und dem produktiven Betrieb der \ac{SaaS}-Lösung in den vergangenen dreieinhalb Jahren, dass besonders für die Bereitstellung einer auf Microservices basierenden Softwarelösung Herausforderungen existieren, welche nicht durch die von der \ac{CF} zur Verfügung stehenden Konzepte und Funktionalitäten gelöst werden können. Aus diesen Gegebenheiten heraus entwickelte sich eine durchaus komplexe Infrastruktur-Landschaft, welche neben den Microservices der eigentlichen Softwarelösung auf einigen zusätzlichen Infrastruktur-Services und weiteren Komponenten basiert.\\
Zudem gestaltet sich vor allem der produktive Betrieb der \ac{SaaS}-Lösung angesichts der limitierten Möglichkeiten zur Konfiguration der \ac{CFAR} sowie den eingeschränkten Funktionalitäten zur automatischen Skalierung der Lösung als schwierig.\\
Aus diesen Gründen verfolgt die vorliegende Thesis die Motivation der Vereinfachung der komplexen Infrastruktur-Landschaft durch die Verwendung der Containerorchestrations-Plattform Kubernetes.
Dies soll besonders mittels des Einsatzes der von der Plattform nativ zur Verfügung stehenden Funktionalitäten und den sich daraus zusätzlich ergebenden Möglichkeiten für den Betrieb der \ac{SaaS}-Lösung erfolgen. 


\newpage
\section{Zielsetzung und Themenabgrenzung}
\label{Themenabgrenzung}
Die Zielsetzung der vorliegenden Thesis ist die Konzeption und prototypische Portierung der \ac{SaaS}-Lösung SAP Subscription Billing auf ein Kubernetes Cluster. Dieser Prototyp soll als Vergleich einer containerbasierten Infrastrukturlösung auf einem Kubernetes Cluster mit der bisherigen Lösung der von der \ac{SCP} angebotenen \ac{CF}-Umgebung dienen. Für den Vergleich der bisherigen auf der \ac{CF} basierenden Infrastrukturlösung mit dem umgesetzten Prototyp soll eine Vergleichsmetrik sowie eine eindeutige Vorgehensweise definiert werden, welche zur Evaluation der beiden Plattformen verwendet werden soll.\\
Die wissenschaftliche Fragestellung der Thesis soll untersuchen, ob sich der Einsatz einer containerbasierten Plattform für die \ac{SaaS}-Lösung eignet und ob diese den Betrieb und die Weiterentwicklung der Software im Vergleich zu den bisherigen Möglichkeiten mit der \ac{CF} verbessern kann. Diese Fragestellung soll mittels der anschließenden Evaluation des Prototyps und des gegenseitigen Vergleiches der Plattformen beantwortet werden können.\\
\\
Der praktische Teil der vorliegenden Thesis wird unter anderem die Formulierung der Vergleichsmetrik und die Analyse der bisherigen Infrastrukturlösung mit der \ac{CF} beinhalten. Das für die Formulierung der Vergleichsmetrik benötigte Wissen soll unter anderem durch die Mitarbeit im agilen Infrastruktur-Team des SAP Subscription Billing Projektes erworben werden. Des Weiteren soll die praktische Portierung der \ac{SaaS}-Lösung auf die containerbasierte Kubernetes Plattform konzeptioniert und prototypisch umgesetzt werden. Der implementierte Prototyp und die von der Kubernetes Plattform nativ unterstützten Funktionalitäten sollen anschließend mit Hilfe der Vergleichsmetrik bewertet werden. Abschließend erfolgt die Zusammenfassung der Evaluationen der beiden Plattformen \ac{CF} und Kubernetes. Zudem soll eine Handlungsempfehlung für die zukünftige technische Architektur des SAP Subscription Billing Projektes ausgesprochen werden.\\
\\
Die vollständige Portierung der Softwarelösung ist nicht Teil der Arbeit und wird ausschließlich mittels der Beschränkung auf einen repräsentativen Teil der Softwarelösung durchgeführt. Außerdem beschränkt sich die Untersuchung auf die von der \ac{SCP} angebotene \ac{CF}-Umgebung, die Containertechnologie Docker und die Container-Orchestrations-Technologie Kubernetes. Ein Vergleich verschiedener \ac{PaaS}- und \ac{IaaS}-Provider ist nicht Teil der vorliegenden Thesis.\\
Darüber hinaus sollte beachtet werden, dass die Evaluation der beiden Plattformen eine Momentaufnahme zum aktuellen Zeitpunkt der Thesis ist, welche durch die Weiterentwicklungen der Plattformen durchaus beeinflusst werden kann. Insgesamt erfolgt der Vergleich und die Evaluation der beiden Plattformen hauptsächlich aus Sicht des Infrastruktur-Teams des SAP Subscription Billing Projektes, welches für die Bereitstellung und den Betrieb der \ac{SaaS}-Lösung verantwortlich ist.

\newpage

\section{Vorgehensweise und Aufbau}
\label{Vorgehensweise}
Der erste Teil der Thesis beinhaltet die theoretische Grundlage mit der Einführung in die Themen Cloud Computing, \ac{SaaS}-Lösungen, Microservices und Containerisierung der Infrastruktur mit Docker und Kubernetes. Zudem erfolgt die Erläuterung des generellen Konzeptes eines Kubernetes Clusters und ein Einblick in die Grundfunktionalitäten der essentiellen Komponenten eines Kubernetes Clusters. Des Weiteren wird ein Überblick in das fachliche Anwendungsgebiet und den technischen Aufbau der SAP Subscription Billing Lösung vorgestellt.\\
Im zweiten Teil der Thesis, welcher die Kapitel 3 und 4 umfasst, wird die innerhalb der vorliegenden Thesis verwendete Vergleichsmetrik formuliert. Zusätzlich erfolgt eine Kategorisierung in funktionale und nicht funktionale Merkmale sowie eine generelle Gewichtung der Kriterien, basierend auf den spezifischen Interessen des Projektes SAP Subscription Billing. Außerdem wird die Vorgehensweise für den Vergleich der beiden Plattformen mit Hilfe der einzelnen Merkmale definiert. Anschließend wird die Bewertung der bisherigen Infrastrukturlösung durchgeführt, welche mit der \ac{CF}-Umgebung auf der \ac{SCP} umgesetzt und implementiert worden ist.\\
Der dritte Teil der Thesis, der aus den Kapiteln 5 und 6 besteht, umfasst die Konzeption der prototypischen Portierung der \ac{SaaS}-Lösung auf ein Kubernetes Cluster. Dabei findet die Auswahl der für den Prototyp verwendeten Microservices und der dafür benötigten weiteren Komponenten statt. Im Übrigen erfolgt die Planung der für die prototypische Portierung relevanten Szenarien für die Bereitstellung unterschiedlicher Landschaften, die automatische Aktualisierung der Anwendung, der Service-to-Service Kommunikation und die Umsetzung der Konzepte für das Monitoring und Logging der Anwendung.\\
Im vierten Teil der Thesis wird die Evaluation des umgesetzten Prototyps und der von der Kubernetes Plattform nativ unterstützten Konzepte und Funktionalitäten mittels der definierten Vergleichsmetrik durchgeführt. Darüber hinaus findet die Vorstellung der Funktionalitäten und den sich daraus ergebenden Möglichkeiten für die Weiterentwicklung und den Betrieb der \ac{SaaS}-Lösung statt, welche innerhalb des Prototyps umgesetzt worden sind.\\
Abschließend werden die Ergebnisse der Evaluationen der beiden Plattformen zusammengefasst. Dabei werden die signifikantesten Erkenntnisse erläutert und miteinander verglichen.\\
Dabei wird auch das Fazit der wissenschaftlichen Untersuchung vorgestellt. Zusätzlich findet ein Ausblick in die zukünftige Infrastrukturlösung des SAP Subscription Billing Projektes statt. Hierbei wird besonders auf eine mögliche Portierung der \ac{SaaS}-Lösung auf ein Kubernetes Cluster eingegangen. Außerdem erfolgt ein Ausblick in die Weiterentwicklung der Kubernetes Plattform und der sich dadurch ergebenden Möglichkeiten für die Entwicklung und den Betrieb von \ac{SaaS}-Lösungen.\\
\\
Innerhalb der Untersuchungsmethodik der wissenschaftlichen Fragestellung wird besonders für den Vergleich der beiden Plattformen eine ausführliche Literaturrecherche durchgeführt. Diese dient zum Verständnis der grundlegenden Konzepte der beiden Plattformen sowie zur Untersuchung der von den Plattformen angebotenen Funktionalitäten.\\
Des Weiteren wird für die praktische Portierung der \ac{SaaS}-Lösung die Methodik des experimentellen Prototypings eingesetzt. Diese Methodik wurde unter anderem aufgrund des Zieles des Aneignens von praktischer Erfahrung mit Kubernetes ausgewählt. Zudem ist besonders für das Infrastruktur-Team des SAP Subscription Billing die praktische Umsetzung eines Prototyps und die anschließende qualitative Evaluation relevant, da dieser als Proof-of-Concept dient und als Grundlage für eine langfristige Portierung der gesamten Infrastrukturlösung eingesetzt werden soll. Insgesamt wird die spezifische Methodik des vertikalen Prototypings eingesetzt, da durch den Prototyp das Zusammenspiel der unterschiedlichen Systemkomponenten mit Hilfe einer repräsentativen Auswahl an Microservices untersucht werden soll.\\


