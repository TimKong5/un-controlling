
\chapter*{Kurzfassung der Bachelorthesis}
\begingroup
\begin{table}[h!]
\setlength\tabcolsep{0pt}
\begin{tabular}{p{3.7cm}p{11.7cm}}
Titel & \DerTitelDerArbeit \\
Verfasser/in: & \DerAutorDerArbeit \\
Kurs: & \DieKursbezeichnung \\
Ausbildungsstätte: & \DerNameDerFirma\\
\end{tabular}
\end{table}
\endgroup
Die vorliegende Thesis beschäftigt sich mit dem Vergleich der beiden Plattformen Cloud Foundry und Kubernetes, welche zur Bereitstellung und dem Betrieb von \ac{SaaS}-Lösungen eingesetzt werden. Ziel der Thesis ist die Evaluation der aktuellen Infrastrukturlösung des SAP Subcription Billing Projektes mittels der hierfür formulierten Vergleichsmetrik. Des Weiteren erfolgt die Konzeption und prototypische Umsetzung der Portierung der Softwarelösung auf ein Kubernetes Cluster. In diesem Zusammenhang werden grundlegende Konzepte einer auf Microservices basierenden Softwarearchitektur, wie beispielsweise die Service-to-Service Kommunikation oder dem Routing der externen Anfragen erstellt und umgesetzt. Des Weiteren erfolgt die Evaluation des Kubernetes Prototyps mit Hilfe der Vergleichsmetrik.\\
Die wissenschaftliche Fragestellung beinhaltet die Untersuchung, ob die Portierung der \ac{SaaS}-Lösung auf ein Kubernetes Cluster als sinnvoll betrachtet werden kann. Der Schwerpunkt der Thesis liegt auf der Konzeption und praktischen Umsetzung des Kubernetes Prototyps sowie der Evaluation der generell unterstützten Funktionalitäten und Möglichkeiten der beiden Plattformen zur Bereitstellung einer Cloud nativen \ac{SaaS}-Lösung.\\
Insgesamt zeigten die durch die Evaluation der beiden Plattformen gewonnen Erkenntnisse und die erfolgreiche Umsetzung des Kubernetes Prototyps, dass die langfristige Portierung der \ac{SaaS}-Lösung auf ein Kubernetes Cluster als sinnvoll betrachtet werden kann. Dabei sind besonders die zusätzlichen Funktionalitäten und Möglichkeiten ausschlaggebend, die Kubernetes zur dynamischen Skalierung der bereitgestellten Anwendungen und zu deren Konfiguration nativ anbietet. Des Weiteren kann durch die Portierung die komplexe Infrastruktur-Landschaft des SAP Subscription Billing Projektes durch das Ersetzen mehrerer aktuell zusätzlich benötigter Infrastruktur-Dienste und Komponenten durch native Funktionalitäten von Kubernetes und dem Service Mesh Istio vereinfacht werden. 
\\
Infolgedessen kann die wissenschaftliche Fragestellung mit einem ``Ja`` beantwortet und die Portierung der \ac{SaaS}-Lösung für das Infrastruktur-Team des SAP Subscription Billing Projektes empfohlen werden. Hierfür dient die vorliegende Thesis und der darin umgesetzte Prototyp unter anderem der Rechtfertigung der Portierung sowie auch als essentielle Grundlage für die Portierung der gesamten Infrastrukturlösung auf ein Kubernetes Cluster.






