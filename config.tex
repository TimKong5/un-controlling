% !TEX root =  master.tex
% 		HYPERREF
%
\usepackage[hyphens]{url}
\usepackage[
hidelinks=true % keine roten Markierungen bei Links
]{hyperref}

% Zwei eigene Befehle zum Setzen von Autor und Titel. Ausserdem werden die PDF-Informationen richtig gesetzt.
\newcommand{\TitelDerArbeit}[1]{\def\DerTitelDerArbeit{#1}\hypersetup{pdftitle={#1}}}
\newcommand{\AutorDerArbeit}[1]{\def\DerAutorDerArbeit{#1}\hypersetup{pdfauthor={#1}}}
\newcommand{\Firma}[1]{\def\DerNameDerFirma{#1}}
\newcommand{\Kurs}[1]{\def\DieKursbezeichnung{#1}}

%For the longtable
\usepackage{longtable}

%		FONT AND INPUT ENCODING
%
\usepackage[T1]{fontenc}
\usepackage[utf8]{inputenc}
%1,5 Zeilenabstand:
\usepackage[onehalfspacing]{setspace}
%		CALCULATIONS
%
\usepackage{calc} % Used for extra space below footsepline
%
%		LANGUAGE SETTINGS
\usepackage{eurosym}
%
\usepackage[ngerman]{babel} 	% German language
\usepackage[german=quotes]{csquotes} 	% correct quotes using \enquote{}

%\usepackage[english]{babel}   % For english language
%\usepackage{csquotes} 	% Richtiges Setzen der Anführungszeichen mit \enquote{}


%		BIBLIOGRAPHY SETTINGS
%
\usepackage[backend=biber, autocite=footnote, style=authoryear, dashed=false]{biblatex} 	%Use Author-Year-Cites with footnotes
% \usepackage[backend=biber, autocite=inline, style=ieee]{biblatex} 	% Use IEEE-Style (e.g. [1])
% \usepackage[backend=biber, autocite=inline, style=alphabetic]{biblatex} 	% Use alphabetic style (e.g. [TGK12])

%%%% APA/Harvard-Style (bitte die nächten zwei Zeilen auskommentieren)
%\usepackage[backend=biber, style=apa]{biblatex} 	
%\DeclareLanguageMapping{german}{german-apa}

\DefineBibliographyStrings{ngerman}{  %Change u.a. to et al. (german only!)
	andothers = {{et\,al\adddot}},
}


%url hard breas
\PassOptionsToPackage{hyphens}{url}\usepackage{hyperref}
%%% Uncomment the following lines to support hard URL breaks in bibliography 
%\apptocmd{\UrlBreaks}{\do\f\do\m}{}{}
%\setcounter{biburllcpenalty}{9000}% Kleinbuchstaben
%\setcounter{biburlucpenalty}{9000}% Großbuchstaben


\setlength{\bibparsep}{\parskip}		%add some space between biblatex entries in the bibliography
\addbibresource{bibliography.bib}	%Add file bibliography.bib as biblatex resource

%Passender Abstand
\usepackage[paper=a4paper,left=20mm,right=30mm,top=25mm,bottom=25mm]{geometry}
%Package um autogenerated Text zu habne
\usepackage{blindtext}
%Worttrennung bei Abkürzung
\hyphenation{I-T}
%		FOOTNOTES 
%
% Count footnotes over chapters
\usepackage{chngcntr}
\counterwithout{footnote}{chapter}

%	ACRONYMS
%%%
%%% WICHTIG: Installieren Sie das neueste Acronyms-Paket!!!
%%%
\makeatletter
\usepackage[printonlyused]{acronym}
\@ifpackagelater{acronym}{2015/03/20}
{%
	\renewcommand*{\aclabelfont}[1]{\textbf{\textsf{\acsfont{#1}}}}
}%
{%
}%
\makeatother

%		LISTINGS
\usepackage{listings}	%Format Listings properly
\renewcommand{\lstlistingname}{Quelltext} 
\renewcommand{\lstlistlistingname}{Quelltextverzeichnis}
\lstset{numbers=left,
	numberstyle=\tiny,
	captionpos=b,
	basicstyle=\ttfamily\small}

%Code Colering
\usepackage[dvipsnames]{xcolor}
\newcommand\YAMLcolonstyle{\color{red}\mdseries}
\newcommand\YAMLkeystyle{\color{black}\bfseries}
\newcommand\YAMLvaluestyle{\color{blue}\mdseries}

\makeatletter

% here is a macro expanding to the name of the language
% (handy if you decide to change it further down the road)
\newcommand\language@yaml{yaml}

\expandafter\expandafter\expandafter\lstdefinelanguage
\expandafter{\language@yaml}
{
	backgroundcolor=\color{white},
	keywords={true,false,null,y,n},
	keywordstyle=\color{darkgray}\bfseries,
	basicstyle=\YAMLkeystyle,                                 % assuming a key comes first
	sensitive=false,
	comment=[l]{\#},
	morecomment=[s]{/*}{*/},
	commentstyle=\color{green}\ttfamily,
	stringstyle=\YAMLvaluestyle\ttfamily,
	moredelim=[l][\color{orange}]{\&},
	moredelim=[l][\color{magenta}]{*},
	moredelim=**[il][\YAMLcolonstyle{:}\YAMLvaluestyle]{:},   % switch to value style at :
	morestring=[b]',
	morestring=[b]",
	basicstyle=\ttfamily\footnotesize,
	breakatwhitespace=false,         
	breaklines=true,                 
	captionpos=b,                    
	keepspaces=true,                 
	numbers=left,                    
	numbersep=5pt,                  
	showspaces=false,                
	showstringspaces=false,
	showtabs=false,                  
	tabsize=2,
	literate =    {---}{{\ProcessThreeDashes}}3
				  {>}{{\textcolor{red}\textgreater}}1     
			      {|}{{\textcolor{red}\textbar}}1 
				  {\ -\ }{{\mdseries\ -\ }}3,
}

% switch to key style at EOL
\lst@AddToHook{EveryLine}{\ifx\lst@language\language@yaml\YAMLkeystyle\fi}
\makeatother

\newcommand\ProcessThreeDashes{\llap{\color{cyan}\mdseries-{-}-}}

%

% Multi line comment
\usepackage{verbatim}
%


%		EXTRA PACKAGES
%\setlength{\footskip}{15mm}
\usepackage{lipsum}    %Blindtext
\usepackage{graphicx} % use various graphics formats
\usepackage[german]{varioref} 	% nicer references \vref
\usepackage{caption}	%better Captions
\usepackage{booktabs} %nicer Tabs
\usepackage{array}
%\usepackage{glossary} 
%\makeglossary 
%\newcolumntype{P}[1]{>{\raggedright\arraybackslash}p{#1}}

%--------------------------------------------------------------------------------
%URL nicht über den Rand
%\usepackage{url}

\setcounter{biburlnumpenalty}{100}
\setcounter{biburlucpenalty}{100}
\setcounter{biburllcpenalty}{100}
%--------------------------------------------------------------------------------
%		ALGORITHMS
\usepackage{algorithm}
\usepackage{algpseudocode}
\renewcommand{\listalgorithmname}{Algorithmenverzeichnis }
\floatname{algorithm}{Algorithmus}


%		FONT SELECTION: Entweder Latin Modern oder Times / Helvetica
\usepackage{lmodern} %Latin modern font
%\usepackage{mathptmx}  %Helvetica / Times New Roman fonts (2 lines)
%\usepackage[scaled=.92]{helvet} %Helvetica / Times New Roman fonts (2 lines)

%		PAGE HEADER / FOOTER
%	    Warning: There are some redefinitions throughout the master.tex-file!  DON'T CHANGE THESE REDEFINITIONS!
\RequirePackage[automark,headsepline,footsepline]{scrpage2}
\pagestyle{scrheadings}
%----------------------------------------------------------------------------
% weniger Abstand vor den Überschriften
\RedeclareSectionCommand[%
beforeskip=0pt,
afterskip=1\baselineskip plus .1\baselineskip minus .167\baselineskip
]{chapter} 
%----------------------------------------------------------------------------
\renewcommand*{\pnumfont}{\upshape\sffamily}
\renewcommand*{\headfont}{\upshape\sffamily}
\renewcommand*{\footfont}{\upshape\sffamily}
\renewcommand{\chaptermarkformat}{}

\clearscrheadfoot

\ifoot[\rule{0pt}{\ht\strutbox+\dp\strutbox}Jonas Breuer]{\rule{0pt}{\ht\strutbox+\dp\strutbox}Jonas Breuer}
\ofoot[\rule{0pt}{\ht\strutbox+\dp\strutbox}\pagemark]{\rule{0pt}{\ht\strutbox+\dp\strutbox}\pagemark}

%Kopf und Fußzeile mehr Abstand


\ohead{\headmark}